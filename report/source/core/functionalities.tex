\subsection{Hamming Code (Error Detection/Correction)}
\emph{Hamming code} is implemented to handle $1-bit$ error detection and correction. Since the message modification can only be of $1$ bit, the usage of hamming code ensures that \textbf{no corrupted frames} are received. The algorithm implementation follows exactly the hamming code algorithm. \emph{Hamming code} is applied on the message payload and the result is padded with zeros to be sent as a \textbf{string of characters}. \\

The implementation of the \emph{hamming code} algorithm can be found in \texttt{src/Node.cc} under two functions :
\begin{itemize}
    \item \texttt{string Node::computeHamming(string s, int \&to\_pad);}
    \begin{minted}{c} 
std::string Node::computeHamming(std::string s, int &to_pad)
{
int str_length = s.length();
int m = 8 * str_length;
int r = 1;
while(m + r + 1 > pow(2 , r))
{
r++;
}
...
}
    \end{minted} 
    \item \texttt{string Node::decodeHamming(string s, int padding);}
    \begin{minted}{c}
std::string Node::decodeHamming(std::string s, int padding)
{
if (!s.size())
{
return "";
}
int str_length = s.length();
int mr = 8 * str_length - padding;
int r = 1;
while(mr + 1 > (pow(2,r)))
{
r++;
}
...
}
    \end{minted}
\end{itemize}

\subsection{Character Count (Framing)}
\emph{Character Count} is implemented as a \textbf{framing method} of the transmitted message. It is applied to the \textbf{message payload} to count characters of the string and \textbf{prepend} the count as one byte to the beginning of the \textbf{message payload}. The \emph{decoding} is done in the same manner. \emph{Character count} framing is applied \textbf{before} hamming code. \\

The implementation of the \emph{character count} algorithm can be found in \texttt{src/Node.cc} under two functions :
\begin{itemize}
    \item \texttt{string Node::addCharCount(string msg);}
    \begin{minted}{c}
std::string Node::addCharCount(std::string msg)
{
uint8_t msg_size = (uint8_t) msg.size() + 1;
std::string framed_msg = "";
framed_msg += msg_size;
framed_msg += msg;
return framed_msg;
}
    \end{minted}
    \item \texttt{bool Node::checkCharCount(string \&msg);}
    \begin{minted}{c}
bool Node::checkCharCount(std::string &msg)
{
if (!msg.size())
{
return false;
}
int msg_size = (int)msg[0] - 1;
msg.erase(msg.begin());
if (msg_size == msg.size())
{
return true;
}
return false;
}
    \end{minted}
\end{itemize}

\subsection{Go Back N (Sliding Window Protocol)}
The used \textbf{data link protocol} is \emph{Go Back N}, where the sender re-transmits the whole current window upon \textbf{acknowledge timeout}. The node can receive a message \emph{(of kind 4)} from the hub, so that it can start talking to another node. Also, the node receives a message \emph{(of kind 5)} to mark the end of the session and it responds with a message \emph{(of kind 4)} to indicate its last acknowledged frame. It can, also, send and receive messages \emph{(of kind 3)} to and from the other node, which contains both data and acknowledges \emph{(piggybacked)}. Finally, the node has two kinds of self-messages. \emph{One} is used for marking \textbf{acknowledge timeout}, so that the \textbf{window pointer} is reset. \emph{Two} is used for sending \textbf{piggybacked messages} at certain intervals. Moreover, the node can have $3$ states :
\begin{enumerate}
    \item \textbf{Active state :} where the node is currently participating in a session.
    \item \textbf{Inactive state :} where the node is not in a session.
    \item \textbf{Dead state :} where the node has no more messages to send.
\end{enumerate}

So, we can summarize our \emph{Go Back N} implementation as follows :
\begin{itemize}
    \item The node receives a message \emph{(of kind 4)} from the hub, so that it can start transmission.
    \item The node sends its first \textbf{piggybacked message} with acknowledge of $-1$ and sends a self-message \emph{(of kind 2)} to schedule \textbf{next message} to be sent and a self-message \emph{(of kind 1)} to set \textbf{acknowledge timeout}.
    \item When the node receives a message from the other node \emph{(of kind 3)}, it decodes the \textbf{incoming frame} to advance its \textbf{excepted frame count}. Also, it decodes the \textbf{incoming acknowledge} to advance \textbf{its window}.
    \item When the node receives self-message \emph{(of kind 1)}, it checks whether the window is advanced. Accordingly, it can reset \textbf{window pointer}.
    \item When the node receives self-message \emph{(of kind 2)}, it sends the next message to the hub.
    \item When the session times out, the node receives an \textit{"end session"} message \emph{(of kind 5)} from the hub. The node turns into \textbf{inactive state} and sends its last acknowledged frame to the hub.
    \item If the last message of the node is acknowledged, the node turns into \textbf{dead state}.
\end{itemize}

The implementation of the \emph{Go Back N} protocol can be found in \texttt{src/Node.cc} under functions :
\begin{itemize}
    \item \texttt{void Node::handleMessage(cMessage *msg);}
    \begin{minted}{c}
void Node::handleMessage(cMessage *msg)
{
if (!this->is_dead)
{
if (msg->isSelfMessage() && !this->is_inactive)
{
switch(msg->getKind()) 
{
case 1:
{
...
}
case 2:
{
...
}
}
}
else
{
if (msg->getKind() == 3 && !this->is_inactive)
{
...
}
else if (msg->getKind() == 4)
{
...
}
else if (msg->getKind() == 5 && !this->is_inactive)
{
...
}
}
}
}
    \end{minted}
    \item \texttt{void Node::sendMsg();}
    \begin{minted}{c}
void Node::sendMsg()
{
if (this->S <= this->Sl)
{
std::string framed_msg = this->addCharCount(this->msgs[this->S]);
const char* msg_payload = framed_msg.c_str();
int padding = 0;
...
}
cMessage * send_next_self_msg = new cMessage("");
send_next_self_msg->setKind(2);
scheduleAt(simTime() + 1 , send_next_self_msg);
}
    \end{minted}
    \item \texttt{void Node::post\_receive\_ack(cMessage *msg);}
    \begin{minted}{c}
void Node::post_receive_ack(cMessage *msg)
{
if (((Imessage_Base *)msg)->getAcknowledge() == -1)
{
return;
}
while(this->Sf <= (((Imessage_Base *)msg)->getAcknowledge()))
{
this->Sf ++;
}
...
}
    \end{minted}
    \item \texttt{void Node::post\_receive\_frame(cMessage *msg);}
    \begin{minted}{c}
void Node::post_receive_frame(cMessage *msg)
{
int received_frame_seq_num = 
((Imessage_Base *)msg)->getSequence_number();
if (received_frame_seq_num == this->R)
{
this->ack = R;
...
bool check = this->checkCharCount(payload);
if (check)
{
EV << " message char count right !" <<endl;
}
else
{
EV << " message char count wrong !" << endl;
}
}
}
    \end{minted}
\end{itemize}

\subsection{Transmission Channel Noise Modelling}
As mentioned, the hub acts as a \textbf{medium controller}, that is why the hub is responsible for adding \textbf{noise} and \textbf{delay} to the transmitted messages. The implemented noise types are \emph{1-bit} \textbf{modification} \emph{(on message payload)}, \textbf{delay}, \textbf{drop} and \textbf{duplication}. The noise is added with \textbf{random probability} and \textbf{multiple types} might be applied on a single message at once. \\

The implementation of the \emph{noise modelling} can be found in \texttt{src/Hub.cc} under function :
\begin{itemize}
    \item \texttt{int Hub::applyNoise(Imessage\_Base *msg);}
    \begin{minted}{c}
int Hub::applyNoise(Imessage_Base * msg)
{
// generate random action from inverted actions 
// ( losing , delaying , none )
int choice =uniform(0,3);
if (choice == 0)
{
EV << " message lost !" << endl;
return 0;
}
...
int prob_modify  = uniform(0,1)*10;
...
if (choice == 2)
{
return 2;
}
else
{
return 3;
}
}
if (choice == 2)
{
return 1;
}
// if none return 4
return 4;
}
    \end{minted}
\end{itemize}

\subsection{Centralized Network Architecture}
Since we are implementing a \textbf{centralized network}, we use a \textbf{hub} to communicate between nodes. The hub is, mainly, responsible for allocating \textbf{sessions} in the following way :
\begin{itemize}
    \item Generate a \textbf{table of pairs} at the beginning of the simulation that includes all the node pairs to communicate.
    \item At \emph{each session time}, the hub starts a new session through sending the two nodes a message \emph{(of kind 4)} to start transmission. The message contains the expected frame to be sent for the opposite nodes.
    \item The hub schedules a self-message \emph{(of kind 1)}, as well, in order to mark the \textbf{beginning} of a new session.
    \item The hub continues to \emph{re-direct} the messages, until the session \emph{times out}.
    \item Once the session times out, the hub sends to the two active nodes an \textit{"end session"} message and receives the last acknowledged frame from each of them.
    \item The hub starts a \emph{new session} between two new nodes and \textbf{ignores} any other messages.
\end{itemize}

The implementation of the \emph{centralized network} can be found in \texttt{src/Hub.cc} under function :
\begin{itemize}
    \item \texttt{void handleMessage(cMessage *msg);}
    \begin{minted}{c}
void Hub::handleMessage(cMessage *msg)
{
if (msg->isSelfMessage())
{
if (msg->getKind() == 1)
{
...
}
else if (msg->getKind() == 2)
{
...
}
else if (msg->getKind() == 6)
{
...
}
}
else
{
if (msg->getKind() == 3)
{
...
}
else if (msg->getKind() == 4)
{
...
}
}
}
    \end{minted}
    \item \texttt{void generatePairs();}
    \begin{minted}{c}
void Hub::generatePairs()
{
int node1, node2;
for(int i =0; i < this->n*this->n; ++i)
{
...
}
}
    \end{minted}
    \item \texttt{void startSession();}
    \begin{minted}{c}
void Hub::startSession()
{
this->indexer ++;
if(this->indexer == this->senders.size())
{
this->indexer = 0;
}
this->sender = this->senders[this->indexer];
this->receiver = this->receivers[this->indexer];
...
}
    \end{minted}
    \item \texttt{void parseMessage(Imessage\_Base *msg);}
    \begin{minted}{c}
void Hub::parseMessage(Imessage_Base * msg)
{
int msg_sender = msg->getSenderModule()->getIndex();
EV << " message sender id = "<< msg_sender << endl;
EV << " hub sender id = " << sender << endl;
EV << " hub receiver id = " << receiver << endl;
if (msg_sender != this->sender && msg_sender != this->receiver)
{
cancelAndDelete(msg);
return;
}
int msg_receiver;
if (msg_sender == this->sender)
{
msg_receiver = this->receiver;
}
else
{
msg_receiver = this->sender;
}
int noise = applyNoise(msg);
int delay = exponential( par("mean_delay").doubleValue());
EV << " delay time = " << delay <<endl;
std::string msg_payload = msg->getMessage_payload();
switch(noise)
{
...
}
this->last_sent_frames[msg_sender] = 
msg->getSequence_number();
}
    \end{minted}
\end{itemize}

\subsection{Statistics Gathering}
The \textbf{statistics gathering} function is implemented inside the hub, since the hub is the one controlling the \textbf{whole transmission process}. Also, the \textbf{transmission noise} is created in the hub. So, the hub keeps track of the \textbf{generated}, \textbf{lost}, \textbf{re-transmitted} and \textbf{duplicated} message for every node. It sets up a self-message \emph{(of kind 2)} to schedule the \textbf{statistics print}. \textbf{Statistics} can be printed for each node \emph{separately} or \emph{collective} for all nodes. So, the following statistics are printed :
\begin{enumerate}
    \item The total number of \textbf{generated} frames.
    \item The total number of \textbf{dropped} frames.
    \item The total number of \textbf{re-transmitted} frames.
    \item The \textbf{percentage} of useful transmitted data \emph{(Efficiency of the system)}.
\end{enumerate}

The implementation of the \emph{statistics gathering} can be found in \texttt{src/Hub.cc} under function :
\begin{itemize}
    \item \texttt{void Hub::initializeStats();}
    \item \texttt{void Hub::updateStats(int node\_idx, int seq\_num, int frame\_size, bool is\_dropped, bool is\_duplicated);}
    \begin{minted}{c}
void Hub::updateStats(int node_idx, int seq_num, int frame_size, 
bool is_dropped, bool is_duplicated)
{
...
if(it != this->generated_frames[node_idx].end())
{
int index = it - this->generated_frames[node_idx].begin();
// increase retransmission times
this->retransmitted_frames[node_idx][index] ++;
// increase drop times (if dropped)
if (is_dropped)
{
this->is_dropped[node_idx][index] ++;
}
// increase duplication times (if duplicated)
if (is_duplicated)
{
this->is_duplicated[node_idx][index] ++;
}
}
else
{
...
}
}
    \end{minted}
    \item \texttt{void Hub::printStats(bool collective);}
    \begin{minted}{c}
void Hub::printStats(bool collective)
{
if (collective)
{
// count all generated frames
int total_generated = 0;
for (int i = 0; i < this->n; i++)
{
total_generated += this->generated_frames[i].size();
}
// count all dropped frames
int total_dropped = 0;
...
// count all retransmitted frames
int total_retransmitted = 0;
for (int i = 0; i < this->n; i++)
{
total_retransmitted += std::accumulate(this->
retransmitted_frames[i].begin(), 
this->retransmitted_frames[i].end(), 0);
}
float useful_data = 0;
float all_data = 0;
for (int i = 0; i < this->n; i++)
{
for (int j = 0; j < this->generated_frames[i].size(); j++)
{
useful_data += this->frame_sizes[i][j];
all_data += (this->retransmitted_frames[i][j] + 
this->is_duplicated[i][j] + 1) * 
(4*3 + this->frame_sizes[i][j]);
}
}
float efficieny = useful_data/all_data;.
...
}
else
{
...
}
}
}
    \end{minted}
\end{itemize}
